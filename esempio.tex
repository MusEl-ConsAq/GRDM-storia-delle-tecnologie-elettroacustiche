\documentclass[a4paper,12pt]{article}
\usepackage{ME_AQ_temp}


%% -------------------------------------- %%
%  - impostare il titolo della tesina in entrambe le righe 
% 
%% -------------------------------------- %%

\newcommand{\mycustomtitle}{Titolo della Tesina}
% Definisci un titolo personalizzato
\newcommand{\setmytitle}[1]{\renewcommand{\mycustomtitle}{#1}}

% Definizione del titolo e dell'autore
\title{\fontsize{14}{17}\bfseries\uppercase{Titolo della Tesina}}
\author{Nome Cognome}
\date{xx/xx/xxxx}

% Sovrascrivi le impostazioni di hyperref per l'indice
\hypersetup{
    linkcolor=black, % Imposta il colore dei link dell'indice a nero
}

\begin{document}

% Pagina 1: Titolo e riassunto
\maketitle
\thispagestyle{empty}

\begin{center}
    \vspace{1cm}
    \textbf{\fontsize{12}{15}\selectfont{Riassunto}}
\end{center}

Testo del riassunto...

% Genera l'indice
\tableofcontents  

% Pagina 2: Introduzione e resto del testo
\newpage

\section{Introduzione}
Testo dell'introduzione...

% Esempi di utilizzo dei comandi personalizzati nel corpo del documento
\section{Esempi di Convenzioni di Redazione}

\subsection{Opere Artistiche e Libri}
Kontakte di \titoloart{Karlheinz Stockhausen}{1964}. La filosofia della musica moderna (\titoloart{Adorno}{1959}).

\subsection{Articoli}
\titoloarticulo{Wiener’s insight into communication} (\titoloart{von Glasersfeld}{1994}).

\subsection{Parole in Lingua Straniera}
Parola in \linguaest{italiano} significa traduzione.

\subsection{Parole Tecniche}
Un'altra parola tecnica: \parolatecnica{analogico}{pino danieleeeeee}
\subsection{Nomi di Persona}
Fino a questo momento non abbiamo citato il nome di \persona{John Cage}{1912}{1992}.

% Esempi di citazioni
\subsection{Citazioni}

\subsubsection{Citazione Breve}
Una citazione breve nel testo: \citazionebreve{Distinguiamo, innanzitutto, i due tipi di segnali con i quali avremo a che fare: segnali analogici e segnali digitali}{Del Duca 1987, p.17}.

\subsubsection{Citazione Estesa}
\citazioneestesa{
    Molti circuiti elettronici analogici, in particolare gli amplificatori, vengono realizzati usando schemi a controreazione allo scopo di ottenere buone prestazioni, migliori che in assenza di reazione. Sebbene il principio della reazione negativa (negative feedback) fosse noto già da tempo (un esempio classico è il regolatore di Watt), la sua introduzione esplicita e la sua formalizzazione viene attribuita all'ingegnere americano Harold S. Black, che negli anni '20 lo utilizzò per risolvere i problemi di stabilità del guadagno e di distorsione negli amplificatori a tubi elettronici per telefonia a grandi distanze. Oltre che negli amplificatori, di cui ci occuperemo in quanto segue, la controreazione trova largo impiego nella strumentazione e nei sistemi di controllo
}{Pallottino 2003}{163}


\subsection{Citazioni}
% Formula numerata
La trasformata di Fourier a tempo discreto (DFT) è definita come:
\begin{equation}
\text{DFT} (x[n]) = X[k] = \sum_{n=0}^{N-1} x[n] e^{-j\frac{2\pi}{N}kn}
\label{eq:dft}
\end{equation}

\section{Sezione Principale}
Testo della sezione principale... 

Esempio di citazione: secondo Einstein \cite{einstein1905}

\subsection{Sottosezione}
Testo della sottosezione...

\footnote{Esempio di nota a piè di pagina.}

\section{Conclusione}

% Aggiungi la bibliografia
\newpage % ---- Inizia una nuova pagina prima della bibliografia
\bibliographystyle{plain}
\bibliography{bibliography}

\end{document}
