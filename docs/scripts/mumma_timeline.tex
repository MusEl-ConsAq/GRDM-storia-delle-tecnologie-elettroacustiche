\documentclass[a4paper,12pt]{article}
\usepackage{chronology}
\usepackage[italian]{babel}
\usepackage{xcolor}

\title{Timeline delle opere di Gordon Mumma}
\author{}
\date{\today}

\begin{document}

\maketitle

\section*{Cronologia delle opere principali}

% Timeline principale con tutti gli eventi
\begin{chronology}[5]{1965}{1973}{\textwidth}[90ex]
\event{1965}{The Dresden Interleaf 13 February 1945}
\event{1966}{Mesa}
\event{1967}{Hornpipe}
\event{1970}{Conspiracy 8}
\event{1971}{Telepos}
\end{chronology}

\vspace{1cm}

\section*{Timeline dettagliata con opere e approcci}

\begin{chronology}[5]{1965}{1973}{\textwidth}[90ex]
\event[1965]{1966}{The Dresden Interleaf 13 February 1945 (deterministico)}
\event[1966]{1967}{Mesa (deterministico)}
\event[1967]{1970}{Hornpipe (generativo)}
\event[1970]{1971}{Conspiracy 8 (generativo)}
\event[1971]{1972}{Telepos (generativo)}
\end{chronology}

\vspace{1cm}

\section*{Timeline delle tecnologie sviluppate}

\begin{chronology}[5]{1965}{1980}{\textwidth}[90ex]
\event[1967]{1975}{Cybersonic Console}
\event[1965]{1972}{Voltage-Controlled Resonant Circuits}
\event[1970]{1980}{Gestural Control Systems}
\end{chronology}

\vspace{1cm}

\section*{Note sulle tecnologie principali}
\begin{itemize}
  \item \textbf{The Dresden Interleaf 13 February 1945}: nastro magnetico, editing di nastro
  \item \textbf{Mesa}: oscillatori, filtri controllati in tensione
  \item \textbf{Hornpipe}: cybersonic console, circuiti di feedback
  \item \textbf{Conspiracy 8}: microfoni, circuiti di feedback
  \item \textbf{Telepos}: cybersonic console, sensori di movimento
\end{itemize}


\section*{Legenda degli approcci}
\begin{itemize}
  \item \textbf{Approccio generativo}: sistemi che reagiscono e evolvono durante l'esecuzione
  \item \textbf{Approccio deterministico}: composizioni con struttura predefinita
\end{itemize}

\end{document}
