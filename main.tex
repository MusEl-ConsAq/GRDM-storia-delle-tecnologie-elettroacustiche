\documentclass[a4paper,12pt]{article}
\usepackage{ME_AQ_temp}
\usepackage{tabularx}

%% -------------------------------------- %%
%  - impostare il titolo della tesina in entrambe le righe 
% 
%% -------------------------------------- %%

\newcommand{\mycustomtitle}{Titolo della Tesina}
% Definisci un titolo personalizzato
\newcommand{\setmytitle}[1]{\renewcommand{\mycustomtitle}{#1}}

% Definizione del titolo e dell'autore
\title{Corsi Accademici di Musica Elettronica DCPL34 Conservatorio A. Casella, L'Aquila \\ \fontsize{14}{17}\bfseries\uppercase{Titolo della Tesina}}
\author{Nome Cognome \\ esame di \bfseries{Storia della Musica Elettroacustica} e \\ \bfseries{Analisi della Musica Elettroacustica}}
\date{xx/xx/xxxx}

% Sovrascrivi le impostazioni di hyperref per l'indice
\hypersetup{
    linkcolor=black, % Imposta il colore dei link dell'indice a nero
}

\begin{document}

% Pagina 1: Titolo e riassunto
\maketitle
\thispagestyle{empty}

\begin{center}
    \vspace{1cm}
    \textbf{\fontsize{12}{15}\selectfont{Sommario}}
\end{center}

Testo del sommario...

\newpage
% Genera l'indice
\tableofcontents  

% Pagina 2: Introduzione e resto del testo
\newpage

% introduzione.tex

\section{Introduzione}
Questo è il testo dell'introduzione al documento.  % Include il file introduzione.tex
% sezione1.tex

\section{Prima Sezione}
Questo è il testo della prima sezione del documento.      % Include il file sezione1.tex
% sezione2.tex

\section{Seconda Sezione}
Questo è il testo della seconda sezione del documento.
      % Include il file sezione2.tex
% conclusione.tex

\section{Conclusione}
Questo è il testo della conclusione del documento.
   % Include il file conclusione.tex


% Aggiungi la bibliografia
\newpage % ---- Inizia una nuova pagina prima della bibliografia
\bibliographystyle{plain}
\bibliography{bibliography}

\end{document}