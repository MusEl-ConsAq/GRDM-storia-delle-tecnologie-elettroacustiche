\documentclass[a4paper,12pt]{article}
\usepackage{ME_AQ_temp}



%% -------------------------------------- %%
%  - impostare il titolo della tesina in entrambe le righe 
% 
%% -------------------------------------- %%

\newcommand{\mycustomtitle}{Titolo della Tesina}
% Definisci un titolo personalizzato
\newcommand{\setmytitle}[1]{\renewcommand{\mycustomtitle}{#1}}

% Definizione del titolo e dell'autore
\title{\fontsize{14}{17}\bfseries\uppercase{Titolo della Tesina}}
\author{Nome Cognome}
\date{xx/xx/xxxx}

% Sovrascrivi le impostazioni di hyperref per l'indice
\hypersetup{
    linkcolor=black, % Imposta il colore dei link dell'indice a nero
}

\begin{document}

% Pagina 1: Titolo e riassunto
\maketitle
\thispagestyle{empty}

\begin{center}
    \vspace{1cm}
    \textbf{\fontsize{12}{15}\selectfont{Riassunto}}
\end{center}

Testo del riassunto...

% Genera l'indice
\tableofcontents  


% Pagina 2: Introduzione e resto del testo
\newpage

\section{Introduzione}
Testo dell'introduzione...

\section{Sezione Principale}
Testo della sezione principale...

esempio di citazione : secondo Einstein \cite{einstein1905}
\subsection{Sottosezione}
Testo della sottosezione...

\footnote{Esempio di nota a piè di pagina.}

\section{conclusione}

% Aggiungi la bibliografia
\newpage % ---- Inizia una nuova pagina prima della bibliografia
\bibliographystyle{plain}
\bibliography{bibliography}



\end{document}
